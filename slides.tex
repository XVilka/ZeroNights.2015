\documentclass[10pt,pdf,utf8,english,compress,hyperref={unicode}]{beamer}
\usepackage{ifxetex}
\usepackage[english,russian]{babel}
\ifxetex
\else
\usepackage[utf8]{inputenc}
\usepackage[T2A]{fontenc}
\usepackage[backend=biber,style=authoryear]{biblatex}
\fi
\usepackage{booktabs}
\usepackage[scale=2]{ccicons}
\usepackage{url}
\usepackage{listings}
\usepackage{booktabs}
\usepackage{hyperref}
\usepackage{drawstack}

\graphicspath{{./pix/}}
%\usetheme[useTitleProgressBar]{m}
\usetheme{m}

% ------ Embed URL in the reference title ------

\ifxetex
% ---- START
\else
\renewbibmacro*{url+urldate}{}

\newbibmacro{string+doiurl}[1]{%
	\iffieldundef{doi}{%
		\iffieldundef{url}{%
			#1%
		}{%
			\href{\thefield{url}}{#1}%
		}%
	}{%
		\href{http://dx.doi.org/\thefield{doi}}{#1}%
	}%
}

\DeclareFieldFormat{title}{\usebibmacro{string+doiurl}{\mkbibemph{#1}}}
\DeclareFieldFormat[article,incollection]{title}{\usebibmacro{string+doiurl}{\mkbibemph{#1}}}

% ------ Nice and shiny bibliography list -------

\setbeamertemplate{bibliography item}{%
	\ifboolexpr{test {\ifentrytype{book}} or test {\ifentrytype{mvbook}}
	or test {\ifentrytype{collection}} or test {\ifentrytype{mvcollection}}
	or test {\ifentrytype{reference}} or test {\ifentrytype{mvreference}} }
	{\setbeamertemplate{bibliography item}[book]}
	{\ifentrytype{online}
		{\setbeamertemplate{bibliography item}[online]}
		{\setbeamertemplate{bibliography item}[article]}}%
	\usebeamertemplate{bibliography item}}

\defbibenvironment{bibliography}
	{\list{}
		{\settowidth{\labelwidth}{\usebeamertemplate{bibliography item}}%
		\setlength{\leftmargin}{\labelwidth}%
		\setlength{\labelsep}{\biblabelsep}%
		\addtolength{\leftmargin}{\labelsep}%
		\setlength{\itemsep}{\bibitemsep}%
		\setlength{\parsep}{\bibparsep}}}
	{\endlist}
	{\item}
% ----- END
\fi

% ------ Document starting here -------


\title{ESIL - Универсальный IL}
\subtitle{ESIL - Intermediate Language для radare2}
\author{Anton Kochkov (@akochkov)}
\date{\today}
\institute{ZeroNights 11-2015}

\ifxetex
\else
\addbibresource{refs.bib}
% Just a hack to workaround 'Package keyvar error: langjapanese undefined'
% Will be fixed in biblatex 3.2 and biber 2.3
% See more at:
% https://tex.stackexchange.com/questions/277314/package-keyval-error-langjapanese-undefined
\NewBibliographyString{langjapanese}
\NewBibliographyString{fromjapanese}
\fi

% ------------------------------------------------------------------------------------
\begin{document}
\maketitle

\begin{frame}[fragile]
  \frametitle{Anton Kochkov}
    \begin{itemize}
    \item Москва, Россия
    \item Хобби - реверс инжиниринг, языки и путешествия
	\item Участник R2 crew и евангелист radare2
    \item ООО Код Безопасности
    \end{itemize}
\end{frame}

% ------------------------------------------------------------------------------------
\section{Краткий обзор Intermediate Languages}

\begin{frame}[fragile]
  \frametitle{Что такое Intermediate Language}
  \begin{itemize}
  \item {\it Intermediate language is the language of an abstract machine designed to aid in the analysis
  of computer programs. \cite{il-wikipedia}}
  \item Используется как в теории (и практике) компиляции
  \item Аналогично незаменим и для декомпиляции
  \item Огромное количество разных академических и практических воплощений
  \item Основа для высокоуровневого анализа - SMT, AEG, AEP, etc
  \end{itemize}
\end{frame}

\begin{frame}[fragile]
  \frametitle{REIL \footfullcite{reil}}
  \begin{itemize}
  \item Изобретен компанией Zynamics
  \item Использовался в продуктах BinNavi, BinDiff
  \item Поддерживает архитектуры x86, ARM, PowerPC
  \item Бесконечная память VM
  \item Бесконечное количество регистров VM
  \item Без Floating Point
  \item Оригинальные утилиты написаны на Java
  \end{itemize}
\end{frame}

\begin{frame}[fragile]
  \frametitle{REIL}
  \begin{itemize}
  \item 17 инструкций
  \item Алиасы для реальных регистров (eax, ebx, r0, \ldots) \footfullcite{reil-zynamics}
  \end{itemize}
  % TODO: Make a table of them
\end{frame}

\begin{frame}[fragile]
  \frametitle{BAP}
  \begin{itemize}
  \item BAP - Binary Analysis Platform \footfullcite{bap-handbook}
  \item Настоящее имя IL - BIL
  \item Развитый фреймворк
  \item Интеграция с другими утилитами - TEMU, libVEX, IDA Pro, Qira, \ldots
  \item Ориентирован на x86, ARM
  \item Без Floating Point
  \end{itemize}
\end{frame}

%
%\begin{frame}[fragile]
%  \frametitle{BAP}
%  \begin{itemize}
%  \item DEMO
%  \end{itemize}
%\end{frame}

\begin{frame}[fragile]
  \frametitle{BitBlaze (VineIL/VEX)}
  \begin{itemize}
  \item BitBlaze \footfullcite{bitblaze} - платформа, аналогичная BAP
  \item Имеет несколько промежуточных языков
  \item VEX IL (libVEX из valgrind) - ``нижний'' уровень
  \item Vine IL - ``верхний'' уровень
  \item Написан на OCaml + C++
  \end{itemize}
\end{frame}

\begin{frame}[fragile]
  \frametitle{VEX IL}
  \begin{itemize}
  \item Явное указание всех side-эффектов для команд
  \item Ближе всего к ESIL
  \item Оттестирован и используется в Valgrind
  \item Хорошо подходит для эмуляции кода
  \item Избыточен
  \end{itemize}
\end{frame}

\begin{frame}[fragile]
  \frametitle{Vine IL \footfullcite{vine-manual} \footfullcite{vine-vuln-analysis}}
  \begin{itemize}
  \item Бесконечная память
  \item Бесконечное количество регистров
  \item Поддержка типов
  \item Поддержка ``variable scope''
  \end{itemize}
\end{frame}

\begin{frame}[fragile]
  \frametitle{RREIL, OpenREIL, MAIL}
  \begin{itemize}
  \item RREIL \footfullcite{rreil} - гибкий язык, замена REIL
  \item RREIL - поддержка типов
  \item RREIL - интересная концепция ``доменов''
  \item MAIL - IL, созданный для анализа Malware
  \item MAIL - позволяет программе перезаписывать себя саму
  \item RREIL и MAIL - опять отсутствие Floating Point
  \end{itemize}
\end{frame}

\begin{frame}[fragile]
  \frametitle{RREIL, OpenREIL, MAIL}
  \begin{itemize}
  \item OpenREIL \footfullcite{openreil} - проект, созданный для использования REIL в современных реалиях
  \item OpenREIL - полноценный фреймворк, как и BAP
  \item OpenREIL отличается от оригинального REIL
  \item Использует libVEX и имеет поддержку SMT-solving
  \end{itemize}
\end{frame}


% -------------------------------------------------------------------------------------
\section{ESIL - сходства и отличия}

\begin{frame}[fragile]
  \frametitle{Краткое описание}
     \begin{itemize}
        \item Evaluable Strings Intermediate Language \footfullcite{r2esil}
		\item Использует обратную польскую нотацию (для скорости)
		\item Не предназначен для чтения человеком
		\item По ``уровню'' приближен к VEX
		\item Небольшое количество инструкций
		\item Полный учёт side-эффектов
      \end{itemize}
\end{frame}

\begin{frame}[fragile]
  \frametitle{Краткое описание}
     \begin{itemize}
        \item Сроектирован для большого количества архитектур
		\item Бесконечная память
		\item Бесконечные регистры
		\item Алиасы (использование ``нативных'' имен)
		\item Есть возможность вызывать куски нативного кода (+syscall)
		\item Возможность добавления ``custom ops''
		\item Нет Floating Point (будет в следующей версии)
      \end{itemize}
\end{frame}

\begin{frame}[fragile]
  \frametitle{Операнды ESIL}
    \begin{table}[H]
	\caption{ESIL Operands \footfullcite{r2esil-instruction-set}}
	\begin{center}
		\resizebox{\linewidth}{!}{%
		\begin{tabular}{|*{4}{c|}}
		\hline
		\textbf{ESIL Opcode} & Operands & Name & Desription\\
		\hline
		\$ & src & Syscall & syscall \\
		\hline
		\$\$ & src & Instruction address & Get address of current instruction \\
		\hline
		== & src,dst & Compare & v = dst - src ; update\_eflags(v) \\
		\hline
		\textless &	src,dst & Smaller &	stack = (dst \textless src) \\
		\hline
		\textless= & src,dst & Smaller or Equal	& stack = (dst \textless= src) \\
		\hline
		\textgreater &	src,dst & Bigger & stack = (dst \textgreater src) \\
		\hline
		|= & src,reg & OR eq & reg = reg | src \\
		\hline
		\end{tabular}}
	\end{center}
	\end{table}
\end{frame}

% -------------------------------------------------------------------------------------
\section{Практическое применение}

\begin{frame}[fragile]
	\frametitle{Radare}
	\begin{figure}
		\includegraphics[scale=0.2]{r2-ubahn.jpg}\footfullcite{r2advert}
	\end{figure}
\end{frame}

\begin{frame}[fragile]
  \frametitle{Radare2 утилиты}
     \begin{itemize}
        \item rax2
        \item rabin2
        \item rasm2
        \item radiff2
        \item rafind2
        \item rahash2
        \item radare2
        \item r2pm
        \item rarun2/ragg2/ragg2-cc
      \end{itemize}
\end{frame}

\begin{frame}[fragile]
  \frametitle{1 command \textless—\textgreater 1 Reverse-Engineering'notion}
  \begin{enumerate}
  \item Каждый символ команды что-то значит \alert{(w = write, p = print)}
  \item Обычно команды - это аббревиатуры действий \alert{pdf = p \textless-\textgreater print d \textless-\textgreater disassemble f \textless-\textgreater function }
  \item Доступна короткая справка для каждой команды \textbf{cmd?}, например \alert{pdf?},\alert{?}, \alert{???}, \alert{???}, \alert{?\$?}, \alert{?@?}
  \end{enumerate}
\end{frame}

\begin{frame}[fragile]
  \frametitle{Radare2 — Основные команды CLI-режима}
  \begin{enumerate}
   \item \alert{\bf{r2 -A}} или \alert{\bf{r2}} + \alert{\bf{aaa}} : Анализ
   \item \alert{\bf{s}} : Переход по указанному адресу
   \item \alert{\bf{pdf}} : Дизассемблирование функции
   \item \alert{\bf{af?}} : Анализ функции
   \item \alert{\bf{ax?}} : Анализ XREF
   \item \alert{\bf{/?}} : Поиск
   \item \alert{\bf{ps?}} : Напечатать строку (print string)
   \item \alert{\bf{C?}} : Комментарии
   \item \alert{\bf{w?}} : Запись (hex, опкодов, etc)
 \end{enumerate}
\end{frame}

\section{Radare2 — Visual mode}
\begin{frame}[fragile]
  \frametitle{Radare2 — Основные команды визуального режима}
  \begin{enumerate}
  \item \alert{\bf{V?}} или просто \alert{\bf{?}} : Помощь по командам
  \item \alert{\bf{p/P}} : переключение между разными визуальными представлениями
  \item Навигация с помощью стрелок/hjkl
  \item \alert{\bf{o}} : переместиться по адресу
  \item \alert{\bf{e}} : визуальный режим настроек
  \item \alert{\bf{v}} : список функций
  \item \alert{\bf{\_}} : HUD
  \item \alert{\bf{V}} : ASCII Graph
  \item \alert{\bf{0-9}} : Прыжок на функцию
  \item \alert{\bf{u}} : Undo
 \end{enumerate}
\end{frame}

\begin{frame}[fragile]
  \frametitle{Эмуляция участков кода}
     \begin{itemize}
        \item \alert{\bf{ae*}} набор инструкций
		\item \alert{\bf{aei}} - инициализация ESIL VM
		\item \alert{\bf{aeim}} - инициализация стека/памяти VM
		\item \alert{\bf{aeip}} - установка IP (Instruction Pointer)
		\item \alert{\bf{aes}} - step в режиме эмуляции ESIL
		\item \alert{\bf{aec[u]}} - continue [until]
		\item \alert{\bf{aef}} - эмуляция функции
      \end{itemize}
\end{frame}

\begin{frame}[fragile]
  \frametitle{Эмуляция участков кода}
     \begin{itemize}
        \item DEMO
      \end{itemize}
\end{frame}

% show demo here - demos/demo5_it8502e
\begin{frame}[fragile]
\ifxetex
  \frametitle{Embedded controller - 8051 - ESIL VM}
\else
  \frametitle{Embedded controller - 8051 - ESIL VM \footfullcite{r2esiltv}}
\fi
  \begin{itemize}
	  \item \alert{\bf{r2 -a 8051 ite\_it8502.rom}}
	  \item \alert{\bf{. ite\_it8502.r2}}
	  \item \alert{\bf{e io.cache=true}} для использования кеширования IO
	  \item запустим \alert{\bf{aei}}
	  \item запустим \alert{\bf{aeim}}
	  \item запустим \alert{\bf{aeip}} для старта с момента указания команды
	  \item \alert{\bf{aecu [addr]}} для эмуляции, пока не достигнем IP = [addr]
  \end{itemize}
\end{frame}
% show demo here
% Reference to the example from here? http://radare.tv/a/44

% Check before the show! Probably won't work
\begin{frame}[fragile]
  \frametitle{Совместная отладка}
     \begin{itemize}
		\item Использование ``подсказок'' ESIL при визуальной отладке
        \item DEMO
      \end{itemize}
\end{frame}

\begin{frame}[fragile]
  \frametitle{Эмуляция VM}
     \begin{itemize}
        \item Позволяет выполнить распаковку или выполнение в VM
		\item Хороший пример - использование ESIL для распаковки Baleful
      \end{itemize}
	  % TODO: Link to the Baleful VM emulation
\end{frame}

\begin{frame}[fragile]
  \frametitle{Автоматическое отображение результатов эмуляции в дизассемблере}
     \begin{itemize}
        \item Отображает в комментариях значения регистров и пямяти
		\item Использует тот же механизм эмуляции кода ESIL VM
		\item Показывает likely/unlikely для условных переходов
		\item \alert{\bf{e asm.emu=true}}
      \end{itemize}
\end{frame}

\begin{frame}[fragile]
  \frametitle{Автоматическое отображение результатов эмуляции в дизассемблере}
     \begin{itemize}
        \item DEMO
      \end{itemize}
\end{frame}

\begin{frame}[fragile]
  \frametitle{Конвертация в другие языки - OpenREIL}
     \begin{itemize}
        \item OpenREIL - развитый фреймфорк
		\item Есть возможность использования SMT
		\item Добавлена возможность конфертации ESIL в OpenREIL
		\item Команда \alert{\bf{aetr}}
      \end{itemize}
\end{frame}

\begin{frame}[fragile]
  \frametitle{Конвертация в другие языки - OpenREIL}
     \begin{itemize}
        \item DEMO
      \end{itemize}
\end{frame}

\begin{frame}[fragile]
  \frametitle{Embedded controller - 8051 - ESIL2REIL}
  \begin{itemize}
	  \item \alert{\bf{r2 -a 8051 ite\_it8502.rom}}
	  \item \alert{\bf{. ite\_it8502.r2}}
	  \item run \alert{\bf{pae 36}} для показа ESIL представления функции 'set\_SMBus\_frequency'
\ifxetex
	  \item \alert{\bf{aetr \`{}pae 36\`}} для конвертации строки ESIL в REIL
\else
	  \item run aetr \`{}pae 36\`{} для конвертации строки ESIL в REIL\footfullcite{openreil}
\fi
	  \item Сохранить вывод в файл и передать управление в OpenREIL
	  \item Можно проделать всё вышеперечисленное с помощью r2pipe скрипта
  \end{itemize}
\end{frame}

% --------------------------------------------------------------------------------------
\section{Radeco IL и radeco decompiler}

\begin{frame}[fragile]
  \frametitle{ESIL -> Radeco \footfullcite{radeco}}
     \begin{itemize}
        \item Использует ESIL в качестве входных данных
		\item Использует другие метаданные из radare2
		\item Соединяется с radare2 через r2pipe
		\item Написан на Rust
      \end{itemize}
\end{frame}

\begin{frame}[fragile]
  \frametitle{Причины появления декомпилятора}
     \begin{itemize}
        \item Существующие FOSS декомпиляторы не учитывают последние исследования
		\item Академические (но интересные) идеи не имеют полноценной реализации
		\item Radare2 нуждается в декомпиляторе
		\item Хорошее и интересное задание для Google Summer of Code
      \end{itemize}
\end{frame}

\begin{frame}[fragile]
  \frametitle{Описание Radeco IL}
     \begin{itemize}
        \item Графовое представление
		\item Взяты идеи из RREIL и MAIL
		\item Использование SSA на этапе лифтинга ESIL -> Radeco IL
		\item Встроенная поддержка DCE (Dead Code Elimination)
		\item Базовая возможность вывода типов \footfullcite{tie}
      \end{itemize}
\end{frame}

\begin{frame}[fragile]
  \frametitle{Radeco demo}
     \begin{itemize}
        \item DEMO
      \end{itemize}
\end{frame}

% --------------------------------------------------------------------------------------
\section{Пути будущего развития}

\begin{frame}[fragile]
  \frametitle{Поддерживаемые архитектуры}
     \begin{itemize}
        \item Сейчас лучше всего поддерживаются x86, ARM, GameBoy, 8051, etc
		\item Глобальная цель - поддержка ESIL для всех архитектур в radare2
		\item Поддержка профилей для выбранных модификаций/вариаций процессоров
      \end{itemize}
\end{frame}

\begin{frame}[fragile]
  \frametitle{Поддерживаемые наборы инструкций}
     \begin{itemize}
        \item Floating point (LLVM/McSema) \footfullcite{floating-point-SO}
		\item Векторные инструкции (SSE, AVX, Neon, etc)
		\item VLIW инструкции (для эмуляции кода DSP)
      \end{itemize}
\end{frame}

\begin{frame}[fragile]
  \frametitle{Визуальная отладка и трассировка}
     \begin{itemize}
        \item Улучшение UI
		\item Возможность визуального сравнения эмуляции и нативного выполнения
		\item Устранение ``мертвого'' кода из ASCII графов на лету
		\item Интеграция в WebUI и Bokken \footfullcite{bokken}
      \end{itemize}
\end{frame}

\begin{frame}[fragile]
  \frametitle{Развитие декомпилятора Radeco}
     \begin{itemize}
        \item Генерация C кода
		\item Поддержка нативных типов
		\item Синхронизация с отладкой
		\item Автовывод типов/распознавание объектов и классов\footfullcite{tie} \footfullcite{il-optimization}
      \end{itemize}
\end{frame}

% ----------------------------------------------------------------------------------------
\ifxetex
\else
\section{References}
\begin{frame}[allowframebreaks]
	\frametitle{A lot of them}
	\printbibliography
\end{frame}
\fi

\end{document}
