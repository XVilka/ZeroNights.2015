\documentclass[10pt,pdf,utf8,english,compress,hyperref={unicode}]{beamer}
\usepackage{ifxetex}
\usepackage[english,russian]{babel}
\ifxetex
\else
\usepackage[utf8]{inputenc}
\usepackage[T2A]{fontenc}
\usepackage[backend=biber,style=authoryear]{biblatex}
\fi
\usepackage{booktabs}
\usepackage[scale=2]{ccicons}
\usepackage{url}
\usepackage{listings}
\usepackage{booktabs}
\usepackage{hyperref}
\usepackage{drawstack}

\graphicspath{{./pix/}}
%\usetheme[useTitleProgressBar]{m}
\usetheme{m}

% ------ Embed URL in the reference title ------

\ifxetex
% ---- START
\else
\renewbibmacro*{url+urldate}{}

\newbibmacro{string+doiurl}[1]{%
	\iffieldundef{doi}{%
		\iffieldundef{url}{%
			#1%
		}{%
			\href{\thefield{url}}{#1}%
		}%
	}{%
		\href{http://dx.doi.org/\thefield{doi}}{#1}%
	}%
}

\DeclareFieldFormat{title}{\usebibmacro{string+doiurl}{\mkbibemph{#1}}}
\DeclareFieldFormat[article,incollection]{title}{\usebibmacro{string+doiurl}{\mkbibemph{#1}}}

% ------ Nice and shiny bibliography list -------

\setbeamertemplate{bibliography item}{%
	\ifboolexpr{test {\ifentrytype{book}} or test {\ifentrytype{mvbook}}
	or test {\ifentrytype{collection}} or test {\ifentrytype{mvcollection}}
	or test {\ifentrytype{reference}} or test {\ifentrytype{mvreference}} }
	{\setbeamertemplate{bibliography item}[book]}
	{\ifentrytype{online}
		{\setbeamertemplate{bibliography item}[online]}
		{\setbeamertemplate{bibliography item}[article]}}%
	\usebeamertemplate{bibliography item}}

\defbibenvironment{bibliography}
	{\list{}
		{\settowidth{\labelwidth}{\usebeamertemplate{bibliography item}}%
		\setlength{\leftmargin}{\labelwidth}%
		\setlength{\labelsep}{\biblabelsep}%
		\addtolength{\leftmargin}{\labelsep}%
		\setlength{\itemsep}{\bibitemsep}%
		\setlength{\parsep}{\bibparsep}}}
	{\endlist}
	{\item}
% ----- END
\fi

% ------ Document starting here -------


\title{ESIL - Универсальный IL}
\subtitle{ESIL - Intermediate Language для radare2}
\author{Anton Kochkov (@akochkov)}
\date{\today}
\institute{ZeroNights 11-2015}

\ifxetex
\else
\addbibresource{refs.bib}
% Just a hack to workaround 'Package keyvar error: langjapanese undefined'
% Will be fixed in biblatex 3.2 and biber 2.3
% See more at:
% https://tex.stackexchange.com/questions/277314/package-keyval-error-langjapanese-undefined
\NewBibliographyString{langjapanese}
\NewBibliographyString{fromjapanese}
\fi

% ------------------------------------------------------------------------------------
\begin{document}
\maketitle

\begin{frame}[fragile]
  \frametitle{Anton Kochkov}
    \begin{itemize}
    \item Moscow, Russia
    \item Love Reverse Engineering, foreign languages and travel
	\item Member of R2 crew, radare2 evangelist
    \item Security Code Ltd.
    \end{itemize}
\end{frame}

% ------------------------------------------------------------------------------------
\section{Intermediate Languages}

\begin{frame}[fragile]
  \frametitle{What is Intermediate Language?}
  \begin{itemize}
  \item {\it Intermediate language is the language of an abstract machine designed to aid in the analysis
  of computer programs. \cite{il-wikipedia}}
  \item Heavily used for academic research and real world tools
  \item Vital for decompilation process
  \item Base for various kind of applications - SMT, AEG, AEP, etc
  \end{itemize}
\end{frame}

\begin{frame}[fragile]
  \frametitle{REIL \footfullcite{reil}}
  \begin{itemize}
  \item Invented by Zynamics company
  \item BinNavi, BinDiff were both based on top of REIL
  \item x86, ARM, PowerPC architectures supported
  \item VM - Infinite memory
  \item VM - Infinite range of registers
  \item Missing Floating Point
  \item Written in Java
  \end{itemize}
\end{frame}

\begin{frame}[fragile]
  \frametitle{REIL}
  \begin{itemize}
  \item 17 instructions
  \item Register aliases (eax, ebx, r0, \ldots) \footfullcite{reil-zynamics}
  \end{itemize}
  % TODO: Make a table of them
\end{frame}

\begin{frame}[fragile]
  \frametitle{BAP}
  \begin{itemize}
  \item BAP - Binary Analysis Platform \footfullcite{bap-handbook}
  \item IL itself called BIL
  \item Well-maintained framework, various tools included
  \item Integration with another tools like: TEMU, libVEX, IDA Pro, Qira, \ldots
  \item Targeted for x86, ARM
  \item Missing Floating Point
  \end{itemize}
\end{frame}

%
%\begin{frame}[fragile]
%  \frametitle{BAP}
%  \begin{itemize}
%  \item DEMO
%  \end{itemize}
%\end{frame}

\begin{frame}[fragile]
  \frametitle{BitBlaze (VineIL/VEX)}
  \begin{itemize}
  \item BitBlaze \footfullcite{bitblaze} - also a platform, like BAP
  \item Uses two intermediate languages
  \item Vine IL - ``low-level'' language
  \item VEX IL (libVEX from valgrind) - ``high-level'' language
  \item Written in OCaml + C++
  \end{itemize}
\end{frame}

\begin{frame}[fragile]
  \frametitle{Vine IL \footfullcite{vine-manual} \footfullcite{vine-vuln-analysis}}
  \begin{itemize}
  \item Implicitly require description of all side-effects
  \item Quite similar to ESIL
  \item Very useful for direct code emulation
  \item Too much unused information
  \end{itemize}
\end{frame}

\begin{frame}[fragile]
  \frametitle{VEX IL}
  \begin{itemize}
  \item Infinite memory
  \item Infinite range of registers
  \item Support types
  \item ``Variable scope'' support
  \item Used in Valgrind and a few other programs
  \item Well-tested and actively maintained
  \end{itemize}
\end{frame}

\begin{frame}[fragile]
  \frametitle{RREIL, OpenREIL, MAIL}
  \begin{itemize}
  \item RREIL \footfullcite{rreil} - modern and flexible alternative to REIL
  \item RREIL - supports types
  \item RREIL - unique conceipt of ``domains''
  \item MAIL - IL, constructed specifically for Malware analysis
  \item MAIL - the only IL, which allows polymorph programs
  \item RREIL and MAIL - both lacks Floating Point support
  \end{itemize}
\end{frame}

\begin{frame}[fragile]
  \frametitle{RREIL, OpenREIL, MAIL}
  \begin{itemize}
  \item OpenREIL \footfullcite{openreil} - reinvention and attempt to spruce up REIL
  \item OpenREIL - self-contained and ready to use framework, like BAP
  \item OpenREIL has major differencies from the original REIL
  \item Based on libVEX and has embedded support of SMT-solving
  \end{itemize}
\end{frame}


% -------------------------------------------------------------------------------------
\section{ESIL - What's different and what's common}

\begin{frame}[fragile]
  \frametitle{Short description}
     \begin{itemize}
        \item Evaluable Strings Intermediate Language \footfullcite{r2esil}
		\item Based on RPN (Reverse Polish Notation)(for the sake of speed)
		\item Designed for evaluation and emulation, not human-reading
		\item Low-level, pretty much alike Vine IL
		\item Small set of the instructions
		\item Implicit specification of all side-effects for each instruction
      \end{itemize}
\end{frame}

\begin{frame}[fragile]
  \frametitle{Short description}
     \begin{itemize}
        \item Designed with a wide range of different architectures in mind
		\item Infinite memory
		\item Infinite set of registers
		\item Register aliases (``native'' names, like ``eax'' or ``cpsr'')
		\item Ability to call external functions (+syscalls)
		\item Ability to implement ``custom ops'' easily
		\item Without Floating Point support (planned, though)
      \end{itemize}
\end{frame}

\begin{frame}[fragile]
  \frametitle{ESIL operands}
    \begin{table}[H]
	\caption{ESIL Operands \footfullcite{r2esil-instruction-set}}
	\begin{center}
		\resizebox{\linewidth}{!}{%
		\begin{tabular}{|*{4}{c|}}
		\hline
		\textbf{ESIL Opcode} & Operands & Name & Desription\\
		\hline
		\$ & src & Syscall & syscall \\
		\hline
		\$\$ & src & Instruction address & Get address of current instruction \\
		\hline
		== & src,dst & Compare & v = dst - src ; update\_eflags(v) \\
		\hline
		\textless &	src,dst & Smaller &	stack = (dst \textless src) \\
		\hline
		\textless= & src,dst & Smaller or Equal	& stack = (dst \textless= src) \\
		\hline
		\textgreater &	src,dst & Bigger & stack = (dst \textgreater src) \\
		\hline
		|= & src,reg & OR eq & reg = reg | src \\
		\hline
		\end{tabular}}
	\end{center}
	\end{table}
\end{frame}

% -------------------------------------------------------------------------------------
\section{Practical usage}

\begin{frame}[fragile]
	\frametitle{Radare}
	\begin{figure}
		\includegraphics[scale=0.2]{r2-ubahn.jpg}\footfullcite{r2advert}
	\end{figure}
\end{frame}

\begin{frame}[fragile]
  \frametitle{Radare2 tools}
     \begin{itemize}
        \item rax2
        \item rabin2
        \item rasm2
        \item radiff2
        \item rafind2
        \item rahash2
        \item radare2
        \item r2pm
        \item rarun2/ragg2/ragg2-cc
      \end{itemize}
\end{frame}

\begin{frame}[fragile]
  \frametitle{1 command \textless—\textgreater 1 Reverse-Engineering'notion}
  \begin{enumerate}
  \item Every character of the command has some meaning \alert{(w = write, p = print)}
  \item Usually they're simple abbreviations \alert{pdf = p \textless-\textgreater print d \textless-\textgreater disassemble f \textless-\textgreater function }
  \item Short usage message for each command can be printed with \textbf{cmd?}, e.g. \alert{pdf?},\alert{?}, \alert{???}, \alert{???}, \alert{?\$?}, \alert{?@?}
  \end{enumerate}
\end{frame}

\begin{frame}[fragile]
  \frametitle{Radare2 — Important commands of CLI-mode}
  \begin{enumerate}
   \item \alert{\bf{r2 -A}} или \alert{\bf{r2}} + \alert{\bf{aaa}} : Анализ
   \item \alert{\bf{s}} : seek to the address or flag
   \item \alert{\bf{pdf}} : print disassembly for function
   \item \alert{\bf{af?}} : perform function analysis
   \item \alert{\bf{ax?}} : do analysis for XREF
   \item \alert{\bf{/?}} : various kinds of search
   \item \alert{\bf{ps?}} : print string
   \item \alert{\bf{C?}} : comments
   \item \alert{\bf{w?}} : write bytes (hex, assembly, etc)
 \end{enumerate}
\end{frame}

\section{Radare2 — Visual mode}
\begin{frame}[fragile]
  \frametitle{Radare2 — Important commands of Visual mode}
  \begin{enumerate}
  \item \alert{\bf{V?}} or just \alert{\bf{?}} : Hotkeys help
  \item \alert{\bf{p/P}} : circle between various visual modes
  \item hjkl/arrows navigation
  \item \alert{\bf{o}} : go to the offset/address
  \item \alert{\bf{e}} : show all config variables
  \item \alert{\bf{v}} : show the functions list
  \item \alert{\bf{\_}} : HUD
  \item \alert{\bf{V}} : ASCII Graph
  \item \alert{\bf{0-9}} : jump to the corresponding function
  \item \alert{\bf{u}} : Undo
 \end{enumerate}
\end{frame}

\begin{frame}[fragile]
  \frametitle{Emulation using ESIL}
     \begin{itemize}
        \item \alert{\bf{ae*}} - evaluate and show r2 commands
		\item \alert{\bf{aei}} - VM initialization
		\item \alert{\bf{aeim}} - VM memory/stack setup
		\item \alert{\bf{aeip}} - to set IP (Instruction Pointer) for VM
		\item \alert{\bf{aes}} - step in ESIL emulation
		\item \alert{\bf{aec[u]}} - continue [until]
		\item \alert{\bf{aef}} - emulate whole function by name
      \end{itemize}
\end{frame}

% show demo here - embed gif/avi/asciinema?
\begin{frame}[fragile]
  \frametitle{Emulation using ESIL}
     \begin{itemize}
        \item DEMO
      \end{itemize}
\end{frame}

% show demo here - demos/demo5_it8502e
\begin{frame}[fragile]
\ifxetex
  \frametitle{Embedded controller - 8051 - ESIL VM}
\else
  \frametitle{Embedded controller - 8051 - ESIL VM \footfullcite{r2esiltv}}
\fi
  \begin{itemize}
	  \item \alert{\bf{r2 -a 8051 ite\_it8502.rom}}
	  \item \alert{\bf{. ite\_it8502.r2}}
	  \item \alert{\bf{e io.cache=true}} to cache IO while emulating
	  \item run \alert{\bf{aei}}
	  \item run \alert{\bf{aeim}}
	  \item run \alert{\bf{aeip}} to start from current IP
	  \item \alert{\bf{aecu [addr]}} to emulate until IP = [addr]
  \end{itemize}
\end{frame}
% show demo here
% Reference to the example from here? http://radare.tv/a/44

% Check before the show! Probably won't work
%\begin{frame}[fragile]
%  \frametitle{Simultaneous debug and emulation}
%     \begin{itemize}
%		\item Использование ``подсказок'' ESIL при визуальной отладке
%        \item DEMO
%      \end{itemize}
%\end{frame}

\begin{frame}[fragile]
  \frametitle{Full fledged emulation in VM}
     \begin{itemize}
        \item Allows to start ESIL VM with predefined properties
		\item Allows to run the code (e.g. unpack) in this VM
		\item Allows to set callbacks on some opcodes
		\item Allows to translate syscalls into real ones (optional)
		\item Good example - unpacking Baleful code \footfullcite{esil-baleful}
      \end{itemize}
\end{frame}

\begin{frame}[fragile]
  \frametitle{Using ESIL emulation in analysis}
     \begin{itemize}
        \item Using emulation of the code to find indirect jumps
		\item Using emulation to find some strings addresses
		\item Started by \alert{\bf{aae}}
		\item \alert{\bf{aae}} is a part of \alert{\bf{aaaa}} command
      \end{itemize}
\end{frame}

\begin{frame}[fragile]
  \frametitle{Using background emulation to improve disassembly output}
     \begin{itemize}
        \item Shows possible registers and memory values in comments
		\item Run ESIL VM with default properties in background
		\item Show ``likely/unlikely'' for conditional jumps
		\item \alert{\bf{e asm.emu=true}}
      \end{itemize}
\end{frame}

\begin{frame}[fragile]
  \frametitle{Using background emulation to improve disassembly output}
     \begin{itemize}
        \item DEMO
      \end{itemize}
\end{frame}

\begin{frame}[fragile]
  \frametitle{Converting into another ILs - OpenREIL}
     \begin{itemize}
        \item OpenREIL - actively maintained framework
		\item Ability to perform symbolic execution using SMT solver
		\item Radare2 has a special command to translate ESIL into OpenREIL
		\item \alert{\bf{aetr}}
      \end{itemize}
\end{frame}

\begin{frame}[fragile]
  \frametitle{Converting into another ILs - OpenREIL}
     \begin{itemize}
        \item DEMO
      \end{itemize}
\end{frame}

\begin{frame}[fragile]
  \frametitle{Embedded controller - 8051 - ESIL2REIL}
  \begin{itemize}
	  \item \alert{\bf{r2 -a 8051 ite\_it8502.rom}}
	  \item \alert{\bf{. ite\_it8502.r2}}
	  \item run \alert{\bf{pae 36}} to show ESIL of the function 'set\_SMBus\_frequency'
\ifxetex
	  \item \alert{\bf{aetr \`{}pae 36\`}} to translate ESIL into REIL
\else
	  \item run aetr \`{}pae 36\`{} to translate ESIL into REIL\footfullcite{openreil}
\fi
	  \item Save this output into the file/pipe and send as OpenREIL input
	  \item Could be easily automated using r2pipe script
  \end{itemize}
\end{frame}

% --------------------------------------------------------------------------------------
\section{Radeco IL and radeco decompiler}

\begin{frame}[fragile]
  \frametitle{ESIL -> Radeco \footfullcite{radeco}}
     \begin{itemize}
        \item Uses ESIL as input
		\item Request more metadata from radare2 (xrefs, functions, etc)
		\item Using r2pipe to talk to radare2
		\item Written in pure Rust
		\item Biggest part of it written during GSoC 2015
		\item Authors are: Sushant Dinesh and David Kreuter \footfullcite{radeco-gsoc}
		\item GSoC 2015 was done under the Openwall's project umbrella
      \end{itemize}
\end{frame}

\begin{frame}[fragile]
  \frametitle{Why Radeco?}
     \begin{itemize}
		\item Existent FOSS decompilers are old and not use modern research
		\item Interesting methods to decompile rarely implemented in FOSS tools
		\item Radare2 as a RE suite need a good decompiler
		\item Challenging still viable task for Google Summer of Code
      \end{itemize}
\end{frame}

\begin{frame}[fragile]
  \frametitle{Radeco IL description}
     \begin{itemize}
        \item Based purely on graphs
		\item Based on some concepts from RREIL and MAIL
		\item Simplification to SSA form while lifting ESIL -> Radeco IL
		\item Automatically performed DCE (Dead Code Elimination)
		\item Types inference \footfullcite{tie}
      \end{itemize}
\end{frame}

\begin{frame}[fragile]
  \frametitle{Radeco demo}
     \begin{itemize}
        \item DEMO
      \end{itemize}
\end{frame}

% --------------------------------------------------------------------------------------
\section{Future improvements}

\begin{frame}[fragile]
  \frametitle{Supported architectures}
     \begin{itemize}
        \item Currently supported: x86, ARM, GameBoy, 8051, etc
		\item Goal is to add support for all architectures in radare2
		\item CPU/SoC/chip profiles for a slight differences between them
      \end{itemize}
\end{frame}

\begin{frame}[fragile]
  \frametitle{Instruction sets}
     \begin{itemize}
        \item Floating point (LLVM/McSema) \footfullcite{floating-point-SO}
		\item SIMD instructions (SSE, AVX, Neon, etc)
		\item VLIW and parallel execution (for DSP architectures emulation)
      \end{itemize}
\end{frame}

\begin{frame}[fragile]
  \frametitle{Visual debugging and tracing}
     \begin{itemize}
        \item General UI improvements
		\item Simple representation of diffs between emulation and native execution
		\item Auto removing dead ways/blocks from ASCII graphs
		\item Integration with WebUI and Bokken \footfullcite{bokken}
      \end{itemize}
\end{frame}

\begin{frame}[fragile]
  \frametitle{Radeco improvements}
     \begin{itemize}
        \item pseudo-C code emission
		\item Wider support for various: native and custom types
		\item Recalculating results on the fly, depending from debugging results
		\item Types inference and function/classes autorecognition\footfullcite{tie} \footfullcite{il-optimization}
      \end{itemize}
\end{frame}

% ----------------------------------------------------------------------------------------
\ifxetex
\else
\section{References}
\begin{frame}[allowframebreaks]
	\frametitle{A lot of them}
	\printbibliography
\end{frame}
\fi

\end{document}
